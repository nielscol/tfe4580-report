Phase locked loops are extraordinarily useful frequency synthesizers that are vital to the operation of virtually all wired and wireless communication systems of today. The trend towards increasingly lower power wireless devices poses an acute need to reduce PLL power consumption. This is a challenge as PLLs typically rank among the highest power consuming components of a radio, and are necessarily so to limit phase noise. Reducing analog PLL power consumption can be a prohibitive challenge as the performance of analog loop filters degrade as a result of unavoidably lower charge pump current. However, recent CMOS process nodes with minimum gate lengths as small as 7nm allow for all-digital PLLs as an attractive alternative to analog designs due to low power consumption associated with the implementation of a digital loop filter. Digital loop-filters have a unique advantage where they can be scaled indefinitely as process nodes advance, while suffering no loss in performance. 

All-digital PLLs introduce new challenges in the process of design in the ultra-low power domain. Low power design is complemented by low complexity design, which in a PLL transcribes to low resolution of phase detectors, digitally tunable oscillators, and loop filter digital data paths. In other words, effects of quantization are strong. Whereas analog designs and high power/resolution digital designs can be effectively analyzed (even by hand) with transfer functions in the Z-domain, time domain simulation is necessary to capture quantization effects in low power, low resolution ADPLLs. This, of course, presents a challenge in manual design of PLLs if simulation is an integral part to the most basic modeling, and thus motivates the creation of an automated design solution. Currently, no PLL design framework is known that automates PLL design for the needs of ultra low power PLLs as characterized here. Of the most prominent pre-existing frameworks, CppSim \hl{[add reference...]} features a utility for design of continuous loop filters, however for this it does not provide any level of performance optimization, nor does it support discrete-time and quantized digital designs. 

Thus in this paper, a new framework written in pure-Python is introduced, which uniquely addresses issues of ultra-low power ADPLL design. Specifically, design of integer-N type PLLs is focused on, as the impetus of this work is an integer-N PLL design project. Topics presented are (a) automatic design and optimization of ADPLL loop filters given target system and component level specs for the PLL, and (b) behavioral time domain PLL simulation for accurate analysis and verification of PLL performance, with an integrated Monte-Carlo sampling variation analysis engine.

A brief outline of the paper is as follows. An introduction to PLL theory is in section \ref{theory}. \hl{Simulation and optimization methods are discussed in section XX}. An example design exercise, a comparison to existing solutions, and general discussion considerations for using the framework are in section \ref{disco}. Finally, section \ref{conclusion} concludes.
\vspace{1em}

The main contributions of this work to PLL design are:
\vspace{-0.8em}
\begin{itemize}
	\setlength\itemsep{-0.8em}
	\item Fully automatic PLL loop filter design and optimization of all digital PLLs from high level specifications in an open source framework.
	\item Integrated behavioral time domain simulation of integer-N digital PLLs for verification of filter and PLL design, with variational analysis of performance and stability.
\end{itemize}
