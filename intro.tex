Phase locked loops are extraordinarily useful frequency synthesizers that are vital to the operation of virtually all wired and wireless communication systems of today. The trend towards increasingly lower power wireless devices poses an acute need to reduce PLL power consumption. This is a challenge as PLLs typically rank among the highest power consuming components of a radio, and are necessarily so to limit oscillator phase noise. A sampling of literature on ultra-low power 2.4GHz radios finds oscillator power consumption as a portion of total radio consumption to be 53\% for the receiver in \cite{regulagadda_2018}, 88\% of the transmitter in \cite{shi_2019}, 52\% of the transmitter in \cite{chen_2019}, and 50\% of the receiver in \cite{pengg_2013}. Reducing analog PLL power consumption can be a prohibitive challenge as the performance of analog loop filters degrade as a result of unavoidably lower charge pump current. However, recent CMOS process nodes with minimum gate lengths as small as 7nm allow for all-digital loop filters and PLLs to be a possible alternative to analog designs due to increasingly low power consumption associated with their implementation. Digital loop-filters have the unique advantage where they can be scaled indefinitely as process nodes advance, suffering no loss in performance, while also having greatly reduced sensitivities to process, voltage, temperature (PVT) variations compared to analog implementations.

All-digital PLLs (ADPLLs) introduce new challenges in the process of design in the ultra-low power domain. Low power design is complemented by low complexity design, which in a PLL transcribes to low resolution of phase detectors, digitally tunable oscillators, and loop filter digital data paths. In other words, effects of quantization are strong. Quantization is an inherently nonlinear process, thus strong quantization is tied to strong nonlinear effects. Consequently, where high resolution digital PLLs with low quantization effects can be effectively analyzed with linear-time invariant transfer functions in the Z-domain, simulation and numerical analysis is necessary to accurately capture quantization effects in low power, low resolution ADPLLs. This, of course, presents a challenge in manual loop filter design of PLLs if simulation is an integral part of the most basic modeling, and thus motivates the creation of an automated design solution. Currently, no openly available PLL design framework automates loop filter design for the needs of ultra low power all-digital PLLs as characterized here. 
% Of the most prominent pre-existing frameworks, CppSim \hl{[add reference...]} features a utility for design of continuous loop filters, however for this it does not provide any level of performance optimization, nor does it support discrete-time and quantized digital designs. 

Thus, in this paper, a new framework, \texttt{pllsim}, written in Python\footnote{Python Software Foundation \url{https://www.python.org/}.} is introduced (this framework is available on GitHub\footnote{\texttt{pllsim} codebase: \url{https://github.com/nielscol/pllsim}.}), which uniquely addresses issues of ultra-low power ADPLL design. Specifically, design of integer-N type PLLs is focused on, as the impetus of this work is an integer-N PLL design project. Topics presented are (a) automatic design and optimization of ADPLL loop filters given target system and component level specifications for the PLL, and (b) behavioral time domain PLL simulation for accurate analysis and verification of loop filter and PLL performance, with an integrated Monte-Carlo sampling variation analysis engine. Due to high phase noise associated with low power design, the optimization approach introduced in this paper focuses on the minimization of total integrated phase noise power to allow for maximum PLL performance on a given power budget.

A brief outline of the paper is as follows. An introduction to PLL theory is in section \ref{theory}. Simulation and optimization methods are discussed in sections \ref{pll_arch}, \ref{methods_lf_design_approach}, \ref{simulator} and \ref{methods_lf_opt}. An example design exercise with the framework, a comparison to existing solutions, and general discussion considerations for using the framework are in section \ref{disco}. Finally, section \ref{conclusion} concludes.
\vspace{1em}

The main contributions of this design framework to PLL design are:
\vspace{-0.8em}
\begin{enumerate}[itemsep=0pt,label=\protect\mycirc{\arabic*}]
	\setlength\itemsep{-0.8em}
	\item Fully automatic loop filter design and optimization for all digital integer-N PLLs in an open source Python framework.
	\item Generation of ready-for-hardware digital filter coefficients from high level PLL specifications (maximum lock time, reference frequency, DCO gain, divider ratio, oscillator phase noise). 
	\item Filter design optimization considering minimization of total phase noise, lock time and quantization errors in the final digital loop filter computed.
	\item An integrated behavioral time domain simulator coupled with the loop filter designer allowing for verification of automatically designed filters for lock time and phase noise performance.
	\item Included parametric sweep and variational analysis in the simulator to verify process tolerance ranges and statistical distributions of PLL performance with process variation.
\end{enumerate}
