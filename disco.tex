\section{Discussion}\label{disco}
Discuss choices for loop filter optimizer (loop filter type)
	how was loop filter transfer function prototype arrived at?
	ie what is the best loop filter design 

	why direct type 1 structure is selected.
	lf noise via simulation with input noise
	optimization of data word precision
	loop filter error due to coefficient round-off


	Why DCO+TDC is primary focus of phase noise optimization (other noise sources are easier to reduce below the limits of these two)
	Plot of BW vs tdc/dco noise, motivating optimum


Talk about why DCO flicker noise doesn't matter (i.e. Reference flicker is much greater)
	Also, we don't optimize for reference flicker, it will be constant no matter what,
	reference is separate from PLL and PLL reponse is flat N multiplier from ref to output.
	Discuss why only random-phase walk component of oscillator noise considered.

Compare to state of art (perrot)
Perrot's pre-existing work: general purpuse simulation architecture, doesn't directly handle optimization for integer-N, especially in heavily quantized case

Models are not accurate for frequencies near or greater than reference frequency???

\hl{Design recommendations:}
\hl{High sampling rate}
\hl{Minimum choice of TDC tesolution, constraints for divider jitter,}

	Recommendations for maximum divider jitter, loop filter resolution
	\subsection{Divider noise constraint}
		Output refered phase noise PSD of TDC:
		\begin{equation}
			S_{\Phi n_{TDC,out}} = \frac{1}{12 f_{ref}}\left|2\pi\frac{N}{M} G(f) \right|^2
		\end{equation}
		Output refered phase noise PSD of divider:
		\begin{equation}
			S_{\Phi n_{div, out}} = f_{ref} \left|2\pi N \sigma_{tn_{div}} G(f)\right|^2
		\end{equation}
		The output-referred phase noise for the TDC and divider have the same frequency dependence. So by setting $S_{\Phi n_{div, out}} < S_{\Phi n_{TDC,out}}$, a constraint to force PLL output divider less than TDC noise can be found:
		\begin{equation}
			\sigma_{tn_{div}} < \frac{1}{\mathrm{M}f_{ref}} = \Delta t_{step_{TDC}}
		\end{equation}
		Must simply ensure that jitter of divider is much less than TDC resolution, which is a reasonable demand. Thus, it is reasonable to ignore divider noise in the phase noise optimization if divider noise can reasonably be made insignificant in the overall output phase noise.

\subsection{Example exercise}
\hl{Use WuRx design specs to motivate design example... put updated/better definition of specs relative to design example}
% \scriptsize
\begin{table}[h!]
	\centering
	\def\arraystretch{1.5}		
	\setlength\arrayrulewidth{0.75pt}
	\setlength{\tabcolsep}{1em} % for the horizontal padding
	\begin{tabular}{|l|r|l|l|}
		\hline 
		\rule[-1ex]{0pt}{2.5ex} \cellcolor{gray!40}\textbf{Parameter} & \cellcolor{gray!40}\textbf{Value} & \cellcolor{gray!40}\textbf{Unit }& \cellcolor{gray!40}\textbf{Notes}\\ 
		\hline 
		\rule[-1ex]{0pt}{2.5ex} \textbf{Frequency}  & 2.4-2.4835 & GHz & 2.4G ISM Band\\ 
		\hline 
		\rule[-1ex]{0pt}{2.5ex} \textbf{Ref. frequency} & 16 & MHz & Yields 6 channels \\ 
		\hline 
		\rule[-1ex]{0pt}{2.5ex} \textbf{Power} & $\leq$ 100  &$\mu$W & \\ 
		\hline 
		\rule[-1ex]{0pt}{2.5ex} \textbf{Residual FM} & $\leq$ 107  &kHz$_{RMS}$ & BER $\leq$ 1e-2, $f_{dev}$=$\pm$250 KHz\\ 
		\hline 
		\rule[-1ex]{0pt}{2.5ex} \textbf{Initial Lock Time} & $\leq$ 50 & $\mu$s & Upon cold start \\ 
		\hline 
		\rule[-1ex]{0pt}{2.5ex} \textbf{Re-lock Time} & $\leq$ 5 & $\mu$s & Coming out of standby \\ 
		\hline 
		\rule[-1ex]{0pt}{2.5ex} \textbf{Bandwidth} & 100 & kHz & (nominally), tunable \\ 
		\hline 
	\end{tabular} 
	% \caption{Assigned specifications for branch line hybrid design.}
	% \label{asgn_specs}
	\caption{System-level specifications}
\end{table}   
% \normalsize

% \scriptsize
\begin{table}[h!]
	\centering
	\def\arraystretch{1.5}		
	\setlength\arrayrulewidth{0.75pt}
	\setlength{\tabcolsep}{1em} % for the horizontal padding
	\begin{tabular}{|l|r|l|l|}
		\hline 
		\rule[-1ex]{0pt}{2.5ex} \cellcolor{gray!40}\textbf{Parameter} & \cellcolor{gray!40}\textbf{Value} & \cellcolor{gray!40}\textbf{Unit }& \cellcolor{gray!40}\textbf{Notes}\\ 
		\hline 
		\rule[-1ex]{0pt}{2.5ex} \textbf{DCO LSB Resolution}  & $\leq$ 50  & kHz & Determined from quantization noise.\\ 
		\hline 
		\rule[-1ex]{0pt}{2.5ex} \textbf{DCO DNL} & $<$ 1 & LSB & Ensures monotonicity \\ 
		\hline 
		\rule[-1ex]{0pt}{2.5ex} \textbf{TDC Resolution} & $\leq$ 3.8  & ns & \\ 
		\hline 
		\rule[-1ex]{0pt}{2.5ex} \textbf{TDC Resolution (bits)} & $\geq$ 4.03 &bits & \\ 
		\hline 
	\end{tabular} 
	\caption{Component-level specifications.}
	% \caption{Assigned specifications for branch line hybrid design.}
	% \label{asgn_specs}
\end{table}   

\FloatBarrier
% \normalsize